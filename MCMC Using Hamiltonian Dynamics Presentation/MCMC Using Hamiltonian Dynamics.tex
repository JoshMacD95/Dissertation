\documentclass{beamer}
%
% Choose how your presentation looks.
%
% For more themes, color themes and font themes, see:
% http://deic.uab.es/~iblanes/beamer_gallery/index_by_theme.html
%
\mode<presentation>
{
  \usetheme{Warsaw}      % or try Darmstadt, Madrid, Warsaw, ...
  \usecolortheme{default} % or try albatross, beaver, crane, ...
  \usefonttheme{default}  % or try serif, structurebold, ...
  \setbeamertemplate{navigation symbols}{}
  \setbeamertemplate{caption}[numbered]
} 
\usepackage{animate}
\usepackage[english]{babel}
\usepackage[utf8]{inputenc}
\title[MCMC using Hamiltonian Dynamics]{Markov Chain Monte Carlo using Hamiltonian Dynamics}
\author{Joshua James MacDonald}
\institute{}
\date{\today}
\begin{document}

% ===============================================

\begin{frame}
  \titlepage
\end{frame}

% Uncomment these lines for an automatically generated outline.
%\begin{frame}{Outline}
%  \tableofcontents
%\end{frame}

% ===============================================

\section{Motivation for HMC}

% ===============================================

\begin{frame}{Performance of MCMC}

\begin{itemize}
\item MCMC is used to simulate s dependent sample from intractable distributions

\item Performance as dimension grows is important with the rise of Big Data

\item How far can we move with each proposal while keeping a optimal acceptance rate

\end{itemize}

\end{frame}


\begin{frame}{Motivation for HMC}

\begin{table}
\centering
\begin{tabular}{|c||c|c|}
\hline
Algorithm & Scaling & Optimal Acceptance Rate $(d \to \infty)$ \\
\hline
RWM & $d^{-1/2}$ & $23.4\%$ (Roberts \& Rosenthal 2001) \\
MALA & $d^{-1/3}$ & $57.4\%$ (Roberts \& Rosenthal 2001) \\
HMC & $d^{-1/4}$ & $64\%$ (Beskos et. al. 2010) \\
\hline
\end{tabular}
\end{table}

\end{frame}

% ===============================================

\section{Hamiltonian Dynamics}

% ===============================================

\begin{frame}{Extended Parameter Space}
\begin{itemize}
\item A ball rolling around a surface
\item At any time the ball has displacement $x$ and momentum $p$
\item Also has a mass $m$
\end{itemize}
\end{frame}
% ==== 
\begin{frame}{The Hamiltonian}
\begin{equation*}
U(x) = - \log \{\pi(x)\}
\end{equation*}

\begin{equation*}
K(p; m) = \frac{1}{2} \frac{p^2}{m}
\end{equation*}

\begin{equation*}
H(x, p; m) = U(x) + K(p;m)
\end{equation*}
\end{frame}

% ==== 
\begin{frame}{Hamiltonian Equations}

\begin{equation*}
\frac{dx}{dt} = \frac{p}{m}
\end{equation*}

\begin{equation*}
\frac{dp}{dt} = - \frac{dU}{dx}
\end{equation*}

Can be generalised to higher dimensions

\begin{equation*}
\frac{dx}{dt} = M^{-1}p
\end{equation*}

\begin{equation*}
\frac{dp}{dt} = - \nabla U(x)
\end{equation*}

\end{frame}
% ==== 

\begin{frame}{1-dimensional Gaussian Example}
% Graph of N(0,1) Density
% Graph of log(N(0,1))
% Graphic of Ball rolling around -log(N(0,1))
\begin{figure}
\centering
\animategraphics[scale = 0.4, loop = true]{30}{Animations/Normal_Potential}{001}{101}
\end{figure}
\end{frame}


\begin{frame}{Why Hamiltonian Dynamics}
\begin{itemize}
\item Reversibility 
\item If Hamiltonian is conserved, we have acceptance probability of 1
\item Volume preservation
\end{itemize}

\end{frame}

% ====

\begin{frame}{Störmer-Verlet Approximation (Leapfrog)}
\begin{itemize}
\item A method of approximating Hamiltonian Dynamics

\item Preserves the desirable properties of Hamiltonian Dynamics

\item Approximately conserves the Hamiltonian

\end{itemize}
\end{frame}


\begin{frame}{Störmer-Verlet Approximation (Leapfrog)}
\begin{itemize}
\item $T$, Integration Time
\item $\varepsilon$, stepsize
\item $L$, Leapfrog steps 
\end{itemize}
\begin{equation}
T = L \varepsilon 
\end{equation}
\end{frame}


% ===============================================

\begin{frame}{Report Writing and Research}

\begin{itemize}
\item Gained experience in report writing throughout my University Career \vskip 5mm

\item Logical structure and clear presentation \vskip 5mm

\item Many of these reports included a Literature review which required research
\end{itemize}

\end{frame}

% ==============================================

\begin{frame}{Communication and Group Work}

\begin{itemize}
\item I am able to convey statistical/mathematical ideas to non-academics \vskip 5mm 

\item I can work effectively as part of a team \vskip 5mm

\end{itemize}


\end{frame}

% ===============================================

\begin{frame}{Computing Experience}

\begin{itemize}
\item My experience is mostly in Object-orientated computer languages \vskip 5mm

\item Model fitting and writing Simulation Algorithms \vskip 5mm

\item Experienced in using \LaTeX{} for reports and presentations \vskip 5mm 

\end{itemize}

\end{frame}

% ===============================================

\begin{frame}{Questions}

\begin{center}
Thank you for listening.
\vskip 0.5cm
Any Questions?

\end{center}

\end{frame}


\end{document}
